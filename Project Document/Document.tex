\documentclass[12pt]{article}
\usepackage{geometry}
\usepackage{hyperref}
\usepackage{graphicx}
\usepackage{titlesec}
\usepackage{fancyhdr}

% 设置页边距
\geometry{a4paper, left=25mm, right=25mm, top=30mm, bottom=30mm}

% 修复 fancyhdr 警告
\setlength{\headheight}{14.49998pt}

% 页眉页脚设置
\pagestyle{fancy}
\fancyhf{}
\fancyhead[L]{COMP3019J University Website Project}
\fancyhead[R]{\thepage}
\fancyfoot[L]{Group 20}
\fancyfoot[R]{October, 2024}

% 标题格式设置
\titleformat{\section}
  {\large\bfseries}{\thesection}{1em}{}
\titleformat{\subsection}
  {\normalsize\bfseries}{\thesubsection}{1em}{}

% 生成可点击链接
\hypersetup{
    colorlinks=true,
    linkcolor=blue,
    filecolor=magenta,      
    urlcolor=cyan,
    pdftitle={COMP 3019J Project},
    pdfpagemode=FullScreen,
}

% 文档标题
\title{COMP3019J University Website Project\\ \Large Design Document}
\date{October 2024}

\begin{document}

% 第一页:标题页
\maketitle
\vspace{2cm}
\begin{figure}[h]  % 插入图片的部分
  \centering
  \includegraphics[width=0.5\textwidth]{title.png}  % 替换成你的图片文件名
\end{figure}
\vspace{6cm}

% 插入作者和日期信息作为文本
\begin{center}
  \textbf{Bohan Zhang 22207251}\\
  \textbf{Yunhan Gao 22207250}\\
  \textbf{Le Liu 22207256}\\
\end{center}

\thispagestyle{empty}
\newpage

% 第二页:目录页
\tableofcontents
\newpage

% 开始正文

\section{Introduction}

\subsection{Project Goals}
The primary goal of this project is to design and implement a comprehensive university website that offers a personalized
 and functional experience for different types of users. The website will provide students, teachers, and other university 
 personnel with an interface to manage their profiles, interact with others, and access university resources based on their 
 roles. The website is expected to enhance user engagement, facilitate communication, and streamline administrative processes 
 within the university. Additionally, the project will focus on secure data handling and innovative functionalities to 
 differentiate the website from existing solutions.

\subsection{Target Users and Use Cases}
The target users of the website include students, teachers, library staff, administrative staff, and security personnel and external visitors.
Each user type will have specific features available to them:
\begin{itemize}
    \item \textbf{Students}: Students will have access to their profiles, which display timetables, dormitory information, 
    and university course. They can register for course, interact with teachers and fellow students, and access academic resources.
    \item \textbf{Teachers}: Teachers will have a profile showing the modules they teach, their office hours, and office location. 
    They will be able to create and manage courses and interact with students.
    \item \textbf{Library staff}: Library staff will have access to features that allow them to manage book inventories, track book 
    loans, and assist users with book reservations. They can also update the library's resources and provide access to digital resources for users.
    \item \textbf{Security personnel}: Security personnel will use the website to monitor campus security issues
    and manage car and e-bikes in campus.
    \item \textbf{Administrative staff}: Administrative users will have full control over managing the system, including overseeing
     user registration and manage the database.
    \item \textbf{External visitors}: Unregistered users will have limited access to certain parts of the website, such as viewing 
    general information about the university, available library resources, and university courses, without the ability to interact with 
    internal functions.
\end{itemize}


\subsection{Technological Stack Overview}
The university website will be built using a modern and robust technological stack to ensure reliability, security, and scalability. 
The core technologies to be used include:
\begin{itemize}
    \item \textbf{Frontend}: The user interface will be developed using HTML, CSS, and JavaScript to create an interactive and
     responsive design. Libraries like Bootstrap or Material UI may be incorporated for consistent styling across the website.
    \item \textbf{Backend}: The server-side logic will be implemented using Flask (Python), providing a flexible framework for 
    managing user authentication, database interactions, and session management.
    \item \textbf{Database}: SQLite will be used as the primary database management system to store user profiles, event data, 
    and other relevant information. The database will include the design of tables with appropriate relationships, such as primary
     and foreign keys, to manage data integrity.
    \item \textbf{Security}: The website will ensure secure handling of sensitive data, such as passwords, using encryption 
    mechanisms. Proper user authentication and authorization will be implemented to protect user privacy and data access.
\end{itemize}
This technology stack has been chosen for its simplicity, flexibility, and suitability for the scale of the university website project, 
ensuring a smooth user experience and maintainability.

\newpage
\section{Website Functionalities}
\subsection{User Registration and Account Management}
% 在这里详细描述用户注册和账号管理功能。
\subsection{User Types and Role-Specific Features}
% 在这里描述不同类型用户及其特定功能。
\subsection{Profile Customization and Display}
% 在这里描述用户个人资料定制和展示功能。
\subsection{User Interaction and Communication}
% 在这里描述用户之间的交互和沟通方式。
\subsection{Unregistered User Access}
% 在这里描述未注册用户的有限访问权限。
\subsection{Security and Data Protection}
% 在这里详细描述安全性和数据保护措施。
\subsection{Administrative Features}
% 在这里描述管理员账户和其功能。
\subsection{Extra Functionality}
% 在这里描述额外的功能。

\newpage
\section{Database Design}
\subsection{Entity-Relationship Diagrams}
% 在这里插入ER图并描述数据库设计。
% \begin{figure}[h]
%     \centering
%     \includegraphics[width=0.8\textwidth]{path_to_your_image.png}
%     \caption{Entity-Relationship Diagram}
% \end{figure}
\subsection{Table Structures and Relationships}
% 在这里描述数据库表结构和关系。
\subsection{Primary and Foreign Key Design}
% 在这里详细描述主键和外键设计。
\subsection{Data Population Strategy}
% 在这里描述数据库如何填充数据。

\newpage
\section{Web Page Design}
\subsection{Mockups and Layout Overview}
% 在这里插入网页设计草图并描述布局。
% \begin{figure}[h]
%     \centering
%     \includegraphics[width=0.8\textwidth]{path_to_your_mockup.png}
%     \caption{Web Page Mockup}
% \end{figure}
\subsection{Color Schemes and Style Guidelines}
% 在这里描述网页的配色方案和风格指南。
\subsection{User Experience Considerations}
% 在这里描述用户体验相关的设计考量。
\subsection{Responsive Design}
% 在这里描述响应式设计方案。

\newpage
\section{Technical Details}
\subsection{File and Directory Structure}
% 在这里描述项目的文件和目录结构。
\subsection{Code Components and Dependencies}
% 在这里描述代码的各个组成部分和依赖关系。
\subsection{API and Database Interaction}
% 在这里描述API与数据库的交互方式。
\subsection{Version Control and Repository Details}
% 在这里描述版本控制系统的使用和代码库的详细信息。

\newpage
\section{Conclusion}
\subsection{Summary of Features}
% 在这里总结网站功能。
\subsection{Future Development Plans}
% 在这里描述未来的开发计划和功能改进。

\end{document}
