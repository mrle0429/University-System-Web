\subsubsection{Course Management}

\paragraph{Description}
Course Management for teachers is a comprehensive use case that enables instructors to create, update, and oversee course information within the university campus management system. This functionality allows teachers to set up new courses, define course objectives, create syllabi, schedule classes, and manage course materials. Teachers can also update course details as the semester progresses, ensuring that students have access to the most current information. This use case is crucial for maintaining an organized and effective academic environment.

\paragraph{Actors}
\begin{itemize}
    \item Teacher
\end{itemize}

\paragraph{Triggers}
\begin{itemize}
    \item New course needs to be created
    \item Existing course information requires updating
    \item Start of a new academic term
    \item Changes in course schedule or content
\end{itemize}

\paragraph{Preconditions}
\begin{itemize}
    \item Teacher is logged into their account
    \item Teacher has the necessary permissions to manage courses
    \item Teacher is on the Course Management section of their dashboard
\end{itemize}

\paragraph{Postconditions}
\begin{itemize}
    \item Course information is successfully created or updated in the system
    \item Students enrolled in the course can view the updated information
    \item System logs the course management activities
\end{itemize}

\paragraph{Data Outcomes}
\begin{itemize}
    \item \textbf{CREATE} - New course entries
    \item \textbf{READ} - Existing course information
    \item \textbf{UPDATE} - Course details, schedules, materials
    \item \textbf{DELETE} - Outdated course materials or information
\end{itemize}

\subsubsection{Student Management}

\paragraph{Description}
Student Management is a crucial use case that allows teachers to oversee and interact with the list of students enrolled in their courses. This functionality enables teachers to view student profiles, track attendance, monitor academic progress, and manage class participation. Teachers can use this feature to identify students who may need additional support, facilitate group assignments, and maintain an organized record of student performance throughout the course. This use case enhances the teacher's ability to provide personalized instruction and support to their students.

\paragraph{Actors}
\begin{itemize}
    \item Teacher
\end{itemize}

\paragraph{Triggers}
\begin{itemize}
    \item Need to view current class roster
    \item Tracking student attendance
    \item Assigning groups for projects
    \item Reviewing student progress in the course
\end{itemize}

\paragraph{Preconditions}
\begin{itemize}
    \item Teacher is logged into their account
    \item Teacher is on the Student Management section of their course dashboard
    \item Course enrollment process is complete
\end{itemize}

\paragraph{Postconditions}
\begin{itemize}
    \item Teacher successfully views or updates student information
    \item Any changes to student records (e.g., attendance) are saved in the system
    \item Student groupings or assignments are updated if applicable
\end{itemize}

\paragraph{Data Outcomes}
\begin{itemize}
    \item \textbf{READ} - Student enrollment list, individual student profiles
    \item \textbf{UPDATE} - Attendance records, student groupings
    \item \textbf{CREATE} - Log entries for student management activities
\end{itemize}

\subsubsection{Grade Entry}

\paragraph{Description}
Grade Entry is a vital use case that enables teachers to input, update, and manage student grades within the university campus management system. This functionality allows teachers to record scores for various assessments, calculate final grades, and provide feedback on student performance. Teachers can enter grades for individual assignments, exams, and participation, which the system can then use to compute overall course grades. This use case ensures accurate and timely recording of student academic performance, facilitating transparent evaluation and progress tracking.

\paragraph{Actors}
\begin{itemize}
    \item Teacher
\end{itemize}

\paragraph{Triggers}
\begin{itemize}
    \item Completion of an assignment or exam
    \item End of a grading period
    \item Need to update previously entered grades
    \item Requirement to provide feedback on student performance
\end{itemize}

\paragraph{Preconditions}
\begin{itemize}
    \item Teacher is logged into their account
    \item Teacher is on the Grade Entry section of their course dashboard
    \item Relevant assignments or exams have been completed by students
\end{itemize}

\paragraph{Postconditions}
\begin{itemize}
    \item Grades are successfully entered or updated in the system
    \item Overall course grades are recalculated if necessary
    \item Students can view their updated grades (subject to release settings)
    \item System logs the grade entry activities
\end{itemize}

\paragraph{Data Outcomes}
\begin{itemize}
    \item \textbf{CREATE} - New grade entries
    \item \textbf{READ} - Existing grade information
    \item \textbf{UPDATE} - Individual assignment grades, overall course grades
    \item \textbf{DELETE} - Incorrect grade entries (if applicable)
\end{itemize}