\subsubsection{Library Search}

\paragraph{Description}
Library Search is a versatile use case that allows various users to explore and locate resources within the university's library system. This functionality enables students, teachers, and guests to search for books, journals, articles, and other academic materials using criteria such as title, author, subject, or keywords. Users can view detailed information about each resource, including its availability status and location. For students and teachers, additional features like checking personal borrowing history may be available. This comprehensive search capability supports academic research, course preparation, and general knowledge exploration for all user types.

\paragraph{Actors}
\begin{itemize}
    \item Student
    \item Teacher
    \item Guest
\end{itemize}

\paragraph{Triggers}
\begin{itemize}
    \item Need to find resources for coursework, research, or teaching
    \item Preparation for assignments, exams, or lectures
    \item General interest in exploring library holdings
    \item Recommendation from colleagues or peers for specific resources
\end{itemize}

\paragraph{Preconditions}
\begin{itemize}
    \item User is on the Library Search section of the system
    \item For students and teachers: User is logged into their account
    \item For guests: Access to public search interface is available
    \item Library database is up-to-date and accessible
\end{itemize}

\paragraph{Postconditions}
\begin{itemize}
    \item User receives search results based on their query
    \item Detailed information about resources is displayed when requested
    \item For logged-in users: System logs the search activity for analytics purposes
    \item For students and teachers: Option to place holds or check personal borrowing history is available
\end{itemize}

\paragraph{Data Outcomes}
\begin{itemize}
    \item \textbf{READ} - Library resource information, availability status, resource location
    \item \textbf{CREATE} - Search history log entry (for logged-in users)
\end{itemize}
\subsubsection{Borrowing Records}

\paragraph{Description}
Borrowing Records is a crucial use case that allows students and teachers to manage and review their library loan history within the university campus management system. This functionality enables users to view their current borrowed items, check due dates, see past borrowing history, and monitor any overdue fines or penalties. Users can also renew eligible items directly through this interface, subject to library policies. The Borrowing Records feature promotes responsible use of library resources and helps users stay organized with their academic materials.

\paragraph{Actors}
\begin{itemize}
    \item Student
    \item Teacher
\end{itemize}

\paragraph{Triggers}
\begin{itemize}
    \item User wants to check current borrowed items
    \item Need to verify due dates for borrowed materials
    \item Desire to review past borrowing history
    \item Intention to renew borrowed items
    \item Checking for any overdue fines or penalties
\end{itemize}

\paragraph{Preconditions}
\begin{itemize}
    \item User is logged into their account
    \item User is on the Borrowing Records section of their dashboard
    \item Library system is up-to-date with the latest borrowing information
\end{itemize}

\paragraph{Postconditions}
\begin{itemize}
    \item User successfully views their borrowing records
    \item Renewal requests (if any) are processed and reflected in the system
    \item System logs the access to borrowing records for security purposes
    \item Any changes to due dates or loan statuses are updated in real-time
\end{itemize}

\paragraph{Data Outcomes}
\begin{itemize}
    \item \textbf{READ} - Current borrowed items, due dates, past borrowing history, fines/penalties
    \item \textbf{UPDATE} - Loan status and due dates (in case of renewals)
    \item \textbf{CREATE} - Log entry of borrowing record access
\end{itemize}

\subsubsection{Profile Management}

\paragraph{Description}
Profile Management is a versatile use case that allows various users to maintain and update their personal information within the university campus management system. This functionality enables students, teachers, librarians, and security personnel to edit various aspects of their profile, including contact details, emergency contacts, and profile pictures. Users can also manage their privacy settings, deciding what information is visible to others. By providing users with control over their personal data, this feature ensures that the university has up-to-date information while respecting individual privacy and accommodating role-specific needs.

\paragraph{Actors}
\begin{itemize}
    \item Student
    \item Teacher
    \item Librarian
    \item Security Personnel
\end{itemize}

\paragraph{Triggers}
\begin{itemize}
    \item User needs to update personal information
    \item Change in contact details or emergency contacts
    \item Desire to modify profile picture
    \item Need to adjust privacy settings
    \item University-wide request for information update
\end{itemize}

\paragraph{Preconditions}
\begin{itemize}
    \item User is logged into their account
    \item User is on the Profile Management section of their dashboard
\end{itemize}

\paragraph{Postconditions}
\begin{itemize}
    \item User's profile information is successfully updated
    \item Changes are immediately reflected in the system
    \item Relevant university departments are notified of significant changes if necessary
\end{itemize}

\paragraph{Data Outcomes}
\begin{itemize}
    \item \textbf{READ} - Current profile information
    \item \textbf{UPDATE} - Personal details, contact information, privacy settings
    \item \textbf{CREATE} - Log entry of profile changes
\end{itemize}

\subsubsection{Forum Participation}

\paragraph{Description}
Forum Participation is an interactive use case that facilitates online discussions within the university's digital platform for various user types. This functionality allows students and teachers to create new discussion threads, reply to existing topics, and interact with each other on various academic and campus-related subjects. The forum serves as a virtual space for knowledge sharing, collaborative learning, and community building. It provides a platform for users to seek advice, share experiences, and stay informed about university events and announcements, while also enabling teachers to guide discussions and provide academic support.

\paragraph{Actors}
\begin{itemize}
    \item Student
    \item Teacher
\end{itemize}

\paragraph{Triggers}
\begin{itemize}
    \item User wants to start a new discussion
    \item User wishes to contribute to an existing thread
    \item Need to seek information or advice from the community
    \item Desire to participate in academic or social conversations
    \item Teacher aims to facilitate course-related discussions
\end{itemize}

\paragraph{Preconditions}
\begin{itemize}
    \item User is logged into their account
    \item User is on the Forum section of the system
    \item User has necessary permissions to post or reply
\end{itemize}

\paragraph{Postconditions}
\begin{itemize}
    \item User's post or reply is successfully added to the forum
    \item Other users can view and respond to the contribution
    \item System updates the forum's activity log
    \item Notifications are sent to relevant users if applicable
\end{itemize}

\paragraph{Data Outcomes}
\begin{itemize}
    \item \textbf{CREATE} - New forum posts or replies
    \item \textbf{READ} - Existing forum threads and replies
    \item \textbf{UPDATE} - User's forum activity status
\end{itemize} 

\subsubsection{Style Customization}

\paragraph{Description}
Style Customization is a user-centric use case that allows various users to personalize the visual appearance of their interface within the university campus management system. This functionality enables students, teachers, librarians, and security personnel to modify aspects of the user interface such as color schemes, layout preferences, and font sizes. By providing this customization option, the system enhances user experience, accommodates individual preferences, and potentially improves accessibility for users with specific visual needs. The ability to customize the interface style helps create a more comfortable and efficient working environment for all users.

\paragraph{Actors}
\begin{itemize}
    \item Student
    \item Teacher
    \item Librarian
    \item Security Personnel
\end{itemize}

\paragraph{Triggers}
\begin{itemize}
    \item User wants to change the visual appearance of their interface
    \item Need to adjust interface for better visibility or accessibility
    \item Desire to personalize user experience
    \item System update introduces new style options
\end{itemize}

\paragraph{Preconditions}
\begin{itemize}
    \item User is logged into their account
    \item User is on the Style Customization section of their settings or preferences
\end{itemize}

\paragraph{Postconditions}
\begin{itemize}
    \item User's interface style preferences are successfully updated
    \item Changes are immediately applied to the user's interface
    \item Style preferences are saved and persist across sessions
\end{itemize}

\paragraph{Data Outcomes}
\begin{itemize}
    \item \textbf{READ} - Current style settings
    \item \textbf{UPDATE} - User's style preferences
    \item \textbf{CREATE} - Log entry of style changes
\end{itemize}